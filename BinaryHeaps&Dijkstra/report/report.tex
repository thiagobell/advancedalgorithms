\documentclass{iiufrgs}
\usepackage[utf8]{inputenc}   % pacote para acentuação
\usepackage{graphicx}           % pacote para importar figuras
\usepackage{times}              % pacote para usar fonte Adobe Times
\usepackage{framed}             % para exemplos e TODOs
\usepackage{biblatex}           % para referências bibliográficas
\usepackage{xcolor}             % cores
\usepackage{hyperref}           % referências
\usepackage{amsmath}
\usepackage{float}

\usepackage{pgfplots}
\usepackage{pgfplotstable}
\usepackage{tikz}

\colorlet{shadecolor}{orange!15}

\title{Laboratório 1}
\author{}{Thiago Bell}

\addbibresource{report.bib}

\begin{document}
\maketitle

\setcounter{chapter}{1}

\section{Tarefa}
Implementar uma \textit{heap} n-ária e usá-la ao implementar o algoritmo de Dijkstra. 
Verificar desempenho da \textit{heap} e do algoritmo de Dijsktra e compará-los com as previs\~oes teóricas.
Os testes presentes no plano de teste sugerido foram utilizados neste trabalho.

\section{Implementaç\~ao}
Os algoritmos foram implementados em C++. A \textit{heap} utiliza um vetor da \textit{standard library} da linguagem para
o armazenamento da árvore de forma eficiente. Os grafos s\~ao representados
por lista de adjacências. Um vetor armazena os vértices, e outro, as arestas.

\section{Ambiente de Teste}
Os experimentos foram realizados usando um processador Intel i7 2600k acompanhado de 8 GiB de RAM. 
O sistema operacional utilizado foi Ubuntu Linux 16.04.

\section{Análise de Complexidade da \textit{Heap}}
A complexidade teórica das operaç\~oes da \textit{heap} foram comparadas com os resultados de experimentos.
Conclui-se que a implementaç\~ao respeita as previs\~oes teóricas.

\subsection{Avaliaç\~ao das Operaç\~oes de Inserç\~ao}
Fixando-se o valor $n = 2^{27}-1$, adicionou-se $n$ elementos na \textit{heap} com chaves no intervalo de $[n,1]$.  
Como se usou uma \textit{heap} de grau 2, isso corresponde a 27 níveis completos na árvore. 
Ao adicionar os elementos, após inserir $2^{i}-1$ chaves, mede-se o tempo $T_i$ e o número de \textit{swaps} 
$I_i$ onde $0<=i<=27$. A complexidade dessa operação é de $O(2^i  i)$.
Com isso pode-se estimar o custo $E = 2^i i$ para cada $i$. Na figura \ref{fig:insert1}
pode-se ver o número de \textit{swaps} e E em funç\~ao de $2^i$. Se essas curvas crescerem da mesma forma,
a complexidade é respeitada. A curva E*15 mostra os valores de $E$ multiplicados por
uma constante mostrando, assim, que assintoticamente o custo é o mesmo.

Na figura \ref{fig:insert2}, mostra-se a raz\~ao entre o tempo obtido experimentalmente 
e o custo teórico. Nota-se que a curva para valores maiores de $2^i$ tende a um valor constante o que 
indicaria que o tempo e o custo est\~ao crescendo na mesma taxa.

Considerando as duas comparaç\~oes, conclui-se que a implementaç\~ao da inserç\~ao na \textit{heap}
respeita a complexidade do algoritmo.


\begin{figure}[H]
\centering
\begin{tikzpicture}

\begin{axis}[
  title={},
  legend style={at={(0.1,0.9)},anchor=north west},
  xlabel=$i$,
  ylabel=$num. op.$]
  ]
\addplot +[mark=none, color=red] table [x=nivel, y=swaps, col sep=comma] {heap_insert.csv};
\addlegendentry{swaps}
\addplot +[mark=none, color=blue] table [x=nivel, y=E, col sep=comma, mark=none, smooth] {heap_insert.csv};
\addlegendentry{$E$}
\addplot +[mark=none, color=green] table [x=nivel, y=Ex15, col sep=comma, mark=none, smooth] {heap_insert.csv};
\addlegendentry{$E*15$}

\end{axis}
\end{tikzpicture}
\caption{Comparaç\~ao entre o número de swaps e a complexidade através do custo esperado ($E$) e seu múltiplo ($E*15$).}
\label{fig:insert1}
\end{figure}

\begin{figure}[H]
\centering

\begin{tikzpicture}

\begin{axis}[
  title={},
  xlabel=$2^i$,
  ylabel=raz\~ao]
  ]
\addplot +[mark=none, color=red] table [x=2nai, y=TE, col sep=comma] {heap_insert.csv};
\end{axis}
\end{tikzpicture}

\caption{Raz\~ao entre tempo de execuç\~ao e o custo esperado.}
\label{fig:insert2}
\end{figure}
\subsection{Avaliaç\~ao das Operaç\~oes de Atualizaç\~ao}
Considerando-se uma variável $i$, $2^i -1$ chaves s\~ao inseridas com valor de $2^i +1$. Em seguida, $2^i$ chaves com valor $2^i +2$ s\~ao adicionadas. Medindo-se o tempo e o número de swaps, as chaves com valor, $2^i +2$ s\~ao atualizadas para valores decrescentes no intervalo $[2^i, 1]$. As operaç\~oes de atualizaç\~ao começam por $2^i$ e a cada operaç\~ao seguinte esse valor é decrescido por 1. 

O custo teórico esperado também é de $E = 2^i \dot i$. Ao comparar com o número de swaps obtido para cada iteraç\~ao, percebeu-se que estes valores s\~ao iguais. Logo, pelo menos em número de swaps, a complexidade é obedecida. No entanto, ao comparar o tempo de execuç\~ao, o resultado obtido foi uma reta inclinada o que indicaria que o tempo de execuç\~ao cresce mais que o esperado.

\begin{figure}[H]
\centering
\begin{tikzpicture}

\begin{axis}[
  title={},
  xlabel=$2^i$,
  ylabel=raz\~ao]
  ]
\addplot +[mark=none, color=red] table [x=2nai, y=TdivE, col sep=comma] {heap_update.csv};
\end{axis}
\end{tikzpicture}
\caption{Mostra a raz\~ao entre o tempo de execuç\~ao e o custo teórico esperado}
\label{fig:update1}
\end{figure}

\subsection{Avaliaç\~ao das Operaç\~oes de Remoç\~ao}
Para um valor $i$, $2^{i} - 1$ chaves com valor aleatórios s\~ao adicionados. Depois, $2^{i - 1}$ chaves s\~ao removidas. O tempo de execuç\~ao e o número de swaps foram medidos para cada i.
O custo teórico dessas operaç\~oes é de $E = 2^i \dot i$. Na figura \ref{fig:delete1} é comparado esse custo com o número de swaps. De forma análoga a feita anteriormente,
também se multiplicou o valor de E por uma constante. A comparaç\~ao do tempo de execuç\~ao com E pode ser vista na figura \ref{fig:delete2}.
A convergência a um valor indica que o tempo de execuç\~ao cresce conforme a complexidade do algoritmo.

\begin{figure}[H]
\begin{tikzpicture}

\begin{axis}[
  legend style={at={(0.1,0.9)},anchor=north west},
  title={},
  xlabel=$i$,
  ylabel=$num. op.$]
  ]
\addplot +[mark=none, color=red] table [x=nivel, y=swaps, col sep=comma] {heap_delete.csv};
\addlegendentry{swaps}
\addplot +[mark=none, color=blue] table [x=nivel, y=E, col sep=comma, mark=none, smooth] {heap_delete.csv};
\addlegendentry{$E$}
\addplot +[mark=none, color=green] table [x=nivel, y=E0.5, col sep=comma, mark=none, smooth] {heap_delete.csv};
\addlegendentry{$E*0.5$}

\end{axis}
\end{tikzpicture}
\centering
\caption{Comparaç\~ao entre o número de swaps e a complexidade através do custo esperado ($E$) e seu múltiplo ($E*0.5$)}
\label{fig:delete1}
\end{figure}

\begin{figure}[H]
\centering
\begin{tikzpicture}

\begin{axis}[
  title={},
  xlabel=$2^i$,
  ylabel=raz\~ao]
  ]
\addplot +[mark=none, color=red] table [x=2nai, y=TdivE, col sep=comma] {heap_delete.csv};

\end{axis}
\end{tikzpicture}
\caption{Mostra a raz\~ao entre o tempo de execuç\~ao e o custo teórico esperado}
\label{fig:delete2}
\end{figure}

\section{Análise de Complexidade do Algoritmo de Dijkstra}
\subsection{Variando o número de Arestas}
Fixando-se o número de vértices em $2^{15}$, variou-se o número de arestas entre $2^{18}$ e $2^{28}$. 
Ao comparar o tempo de execuç\~ao com o custo teórico de $E = (n+m)*log(n)$, 
obteve-se uma convergência a um valor constante como pode ser visto na figura \ref{fig:dij_vertex}. 
Os números de \textit{inserts} e \textit{deletes} foram
menores ou iguais ao de vértices e o de \textit{updates} menor ou igual ao de arestas. Uma regress\~ao
linear na figura \ref{fig:dij_vertex_linear_regression} sobre o os dados obtidos mostrou que o tempo de execução
cresce exponencialmente de acordo com a soma do número de arestas.

\begin{figure}[H]
\centering

\begin{tikzpicture}

\begin{axis}[
  title={},
  xlabel=$m$,
  ylabel=$T/(n+m)log(n)$]
  ]
\addplot +[mark=none, color=red] table [x=m, y=tratio, col sep=comma, mark=none, smooth] {fixed_vertex.csv};
\end{axis}
\end{tikzpicture}
\caption{Mostra a raz\~ao entre o tempo de execuç\~ao e o custo teórico esperado}
\label{fig:dij_vertex}
\end{figure}


\begin{figure}[H]
\centering

\begin{tikzpicture}

\begin{axis}[
  title={},
  xlabel=$log(m)$,
  ylabel=$log(time)$]
  ]
\addplot +[mark=o, color=red,only marks] table [x=logm, y=logtime, col sep=comma] {fix_vertex_lin_regression.csv};
\addplot +[mark=none, color=blue] table [x=logm, y=pred, col sep=comma, mark=none, smooth] {fix_vertex_lin_regression.csv};
\end{axis}
\end{tikzpicture}
\caption{Regress\~ao linear. O número de vértices foi desconsiderado pois é constante.}
\label{fig:dij_vertex_linear_regression}
\end{figure}

\subsection{Variando o número de Vértices}
Fixou-se o número de arestas em $2^{20}$ e variou-se o número de vértices entre $2^{11}$ e $2^{18}$. O custo teórico é o mesmo do caso anterior,
$E = (n+m)*log(n)$. Encontrou-se um problema onde o grafo se tornavo excessivamente esparso e o algoritmo de dijkstra encerrava sem percorrer 
parte significante do grafo. Poucas amostras foram obtidas como fica evidente na figura \ref{fig:dij_edge}.

\begin{figure}[H]
\begin{tikzpicture}

\begin{axis}[
  title={},
  xlabel=$n$,
  ylabel=$T/(n+m)log(n)$]
  ]
\addplot +[mark=none, color=red] table [x=n, y=TdivE, col sep=comma, mark=none, smooth] {fixed_edge.csv};
\end{axis}
\end{tikzpicture}
\centering
\caption{Mostra a raz\~ao entre o tempo de execuç\~ao e o custo teórico esperado}
\label{fig:dij_edge}
\end{figure}

\begin{figure}[H]
\begin{tikzpicture}

\begin{axis}[
  title={},
  xlabel=$log(n)$,
  ylabel=$log(time)$]
  ]
\addplot +[mark=o, color=red,only marks] table [x=logn, y=logtime, col sep=comma] {fix_edge_lin_regression.csv};
\addplot +[mark=none, color=blue] table [x=logn, y=pred, col sep=comma, mark=none, smooth] {fix_edge_lin_regression.csv};
\end{axis}
\end{tikzpicture}
\centering
\caption{Regress\~ao linear. O número de arestas foi desconsiderado pois é constante.}
\label{fig:dij_edge_lin_reg}
\end{figure}

\section{Verificando a implementaç\~ao de Dijkstra com grandes instâncias}
Testou-se a implementaç\~ao do algoritmo com duas instâncias conforme recomendado no plano de teste. Para o caso de teste consistindo da rede completa dos Estados Unidos provida na definiç\~ao do trabalho, obteve-se um tempo médio de execuç\~ao de 169 segundos e consumo de memómia máximo de 3.64 GB. Para a rede de Nova York, foram 1.83 segundos e 43.7 MB de memória.


\section{Testando Grau Ótimo da Heap}
Testou-se a implementaç\~ao da heap variando-se o grau da mesma e usando as redes de Nova York e dos Estados Unidos assim como outras geradas 
aleatóriamente geradas com uma vers\~ao modificada do algoritmo gerador oferecido. Um comportamento interessante foi obtido. 
Para as redes reais, os resultados foram melhores para graus 4 e 8. Entretando, para as artificiais a tendência foi de quanto maior o grau da heap,
melhor o desempenho ao executar o algoritmo de Dijkstra. A figura \ref{fig:narity} mostra os resultados. Os tempos de execuç\~ao de cada grafo
foram normalizados em relaç\~ao ao tempo médio obtido para aquela rede. Acredita-se que esse comportamento, possa ser explicado pela aleatoriedade
na atribuiç\~ao de grafos o que resulta numa topologia diferente daquelas encontradas em redes reais.
\begin{figure}[H]
\centering
\begin{tikzpicture}

\begin{axis}[
  title={},
  legend style={at={(1.1,1)},anchor=north west},
  xtick={2,4,8,16,32,64},
  xlabel=grau,
  ylabel=tempo normalizado]
  ]
\addplot +[mark=o, color=red] table [x=grau, y=NY, col sep=comma, smooth] {heap_narity.csv};
\addlegendentry{NY}
\addplot +[mark=o, color=green] table [x=grau, y=USA, col sep=comma, smooth] {heap_narity.csv};
\addlegendentry{USA}
\addplot +[mark=o, color=blue] table [x=grau, y=2na15, col sep=comma, smooth] {heap_narity.csv};
\addlegendentry{$2^{15}$}
\addplot +[mark=o, color=purple] table [x=grau, y=2na16, col sep=comma, smooth] {heap_narity.csv};
\addlegendentry{$2^{16}$}
\addplot +[mark=o, color=black] table [x=grau, y=2na17, col sep=comma, smooth] {heap_narity.csv};
\addlegendentry{$2^{17}$}
\end{axis}
\end{tikzpicture}

\caption{Tempo normalizado de execuç\~ao do algoritmo de Dijkstra}
\label{fig:narity}
\end{figure}

\section{Conclus\~ao}
Excluindo-se o caso do tempo de execuç\~ao das atualizaç\~oes quando comparado com a expectativa teórica, as implementaç\~oes dos 
algoritmos respeitaram a complexidade teórica dos mesmos. Procurou-se, também, o grau da heap com o valor mais adequado ao algoritmo de Dijkstra. Chegou-se
a conclus\~ao que a rede com a qual se testa pode influenciar os resultados desse teste.

\end{document}