\documentclass{iiufrgs}
\usepackage[utf8]{inputenc}   % pacote para acentuação
\usepackage{graphicx}           % pacote para importar figuras
\usepackage{times}              % pacote para usar fonte Adobe Times
\usepackage{framed}             % para exemplos e TODOs
\usepackage{biblatex}           % para referências bibliográficas
\usepackage{xcolor}             % cores
\usepackage{hyperref}           % referências
\usepackage{amsmath}
\usepackage{float}

\usepackage{pgfplots}
\usepackage{pgfplotstable}
\usepackage{tikz}

\colorlet{shadecolor}{orange!15}

\title{Laboratório 5 - Algoritmo de Cristofides}
\author{}{Thiago Bell}

\addbibresource{report.bib}

\begin{document}
\maketitle

\setcounter{chapter}{1}

\section{Tarefa}
Implementar o algoritmo de Cristofides para o problema do caixeiro viajante.

\section{Implementaç\~ao}
O algoritmo foi implementado em C++. A implementação de grafo é feita através do
armazenamento de cada aresta em uma estrutura de dados especial. Em diferentes 
momentos da execução do algoritmo é necessário acessar essa informação de formas
diferentes. No cálculo da \textit{minimum spanning 
tree} precisa-se ordenar as arestas por comprimento e no pareamento precisa-se
acessá-las a partir dos seus respectivos vértices. Por isso, armazena-se listas de ponteiros.
Uma lista de arestas para cada vértice utilizando ponteiros também é utilizada. 
Assim, pode-se ordenar os ponteiros e manter a ordem anterior mantendo válida 
outras referências a esses dados.

Cada estrutura de arestas, representava uma ou mais arestas nao direcionadas entre dois vértices.
Um campo númerico determinava o número de arestas paralelas entre os vértices. No caso de uma aresta
estar na \textit{minimum spanning tree} e também ser selecionada no pareamento esse número passa a ser dois.

O algoritmo usado para o cálculo da \textit{minimum spanning tree} foi o de Kruskal com complexidade $O(log n)$
onde $n$ é o número de vértices. Para o ciclo euleriano, foi usado o algoritmo de Hierholzer com custo $O(m)$.



\section{Ambiente de Teste}
Os experimentos foram realizados usando um processador Intel  i7-2600k 
acompanhado de 8 GiB de RAM. 
O sistema operacional utilizado foi Ubuntu Linux.

\section{Testes Realizados}
Testes foram realizados com as redes sugeridas. Os tempos de execução foram medidos. Os resultados foram comparados com os caminhos ótimos apresentados na TSPLIB.


\section{}



\section{Conclus\~ao}
Implementou-se o algoritmo de Hopcroft-Karp . Verificou-se que a implementação respeita a
complexidade do algoritmo. Além disso, esse algoritmo tem melhor performance que o de Fork-Fulkerson com uma mesma instância reduzida ao problema de fluxo máximo.
\end{document}
