\documentclass{iiufrgs}
\usepackage[utf8]{inputenc}   % pacote para acentuação
\usepackage{graphicx}           % pacote para importar figuras
\usepackage{times}              % pacote para usar fonte Adobe Times
\usepackage{framed}             % para exemplos e TODOs
\usepackage{biblatex}           % para referências bibliográficas
\usepackage{xcolor}             % cores
\usepackage{hyperref}           % referências
\usepackage{amsmath}
\usepackage{float}

\usepackage{pgfplots}
\usepackage{pgfplotstable}
\usepackage{tikz}

\colorlet{shadecolor}{orange!15}

\title{Laboratório 3 - Maxflow}
\author{}{Thiago Bell}

\addbibresource{report.bib}

\begin{document}
\maketitle

\setcounter{chapter}{1}

\section{Tarefa}
Implementar o algoritmo de Ford-Fulkerson usando a estratégia de \emph{fattest path} para o problema de fluxo máximo. 

\section{Implementaç\~ao}
Os algoritmos foram implementados em C++. A implementação de grafo usa listas de adjacências. Cada aresta armazena informações de
capacidade e fluxo atual. os valores dos \emph{forward edges} e \emph{backward edges} são calculados baseados nessas informações.

\section{Ambiente de Teste}
Os experimentos foram realizados usando um processador Intel i7 2600k acompanhado de 8 GiB de RAM. 
O sistema operacional utilizado foi Ubuntu Linux 16.10.

\section{Análise de Complexidade do Algoritmo de Ford-Fulkerson}
A complexidade teórica do algoritmo foi comparado com resultados experimentais.
Conclui-se que a implementaç\~ao respeita as previs\~oes teóricas.

\subsection{Complexidade}
A complexidade do algoritmo depende do número de iterações que ele executa. Isso é, o número de vezes em que ele incrementa o fluxo.
Supondo um limitante superior $I$ para o número de iterações, a complexidade do algoritmo é de:

\begin{equation*}
\label{eq:ql}
O((nlogn + m)I)
\end{equation*}

\subsection{Formulação do Teste}
Para testar o algoritmo gerou-se redes mesh com o gerador de \emph{Washington} sempre mantendo o número de colunas igual ao de 
linhas. Executou-se o algoritmo para estas redes e comparou-se o tempo de execução com a previsão teórica. Para calcular-se o custo
para uma determinada instância usou-se o número de iterações do algoritmo ao executar como limitante superior $I$.

\subsection{Experimento e Comparação}
Para redes com dimensões laterais variando de 100 até 500 em incrementos de 100, mediu-se o tempo de execução. A comparação destes tempos
com o custo teórico pode ser vista na figura \ref{fig:comparison}. Percebe-se que para todos os casos a razão entre esses dois valores permaneceu
razoavelmente constante. Dessa forma, a implementação segue a complexidade do algoritmo.

\begin{figure}[H]
\centering
\begin{tikzpicture}
\begin{axis}[
  title={},
  ymin = 0,
  xlabel=dimens\~ao,
  ylabel=raz\~ao],
\addplot +[mark=o, color=red] table [x=dimension, y=ratio, col sep=comma] {data.csv};
\end{axis}
\end{tikzpicture}
\caption{Mostra a raz\~ao entre o custo teórico esperado e o tempo de execuç\~ao}
\label{fig:comparison}
\end{figure}


\section{Conclus\~ao}
Implementou-se o algoritmo de Ford-Fulkerson com a estratégia do \emph{Fattest-Path}. Verificou-se que a implementação respeitava a
complexidade do algoritmo.
\end{document}
