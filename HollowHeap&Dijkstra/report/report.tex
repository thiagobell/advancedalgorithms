\documentclass{iiufrgs}
\usepackage[utf8]{inputenc}   % pacote para acentuação
\usepackage{graphicx}           % pacote para importar figuras
\usepackage{times}              % pacote para usar fonte Adobe Times
\usepackage{framed}             % para exemplos e TODOs
\usepackage{biblatex}           % para referências bibliográficas
\usepackage{xcolor}             % cores
\usepackage{hyperref}           % referências
\usepackage{amsmath}
\usepackage{float}

\usepackage{pgfplots}
\usepackage{pgfplotstable}
\usepackage{tikz}

\colorlet{shadecolor}{orange!15}

\title{Laboratório 1 - Hollow Heaps e Dijkstra}
\author{}{Thiago Bell}

\addbibresource{report.bib}

\begin{document}
\maketitle

\setcounter{chapter}{1}

\section{Tarefa}
Implementar uma \textit{Hollow Heap} e usá-la ao implementar o algoritmo de Dijkstra. 
Verificar desempenho da \textit{Hollow Heap} e do algoritmo de Dijsktra e compará-los com as previs\~oes teóricas.
Os testes presentes no plano de teste sugerido foram utilizados neste trabalho.

\section{Implementaç\~ao}
Os algoritmos foram implementados em C++. A \textit{Hollow Heap} utiliza memória alocada dinamicamente para armazenar os seus nodos.
Os grafos s\~ao representados por lista de adjacências. Um vetor armazena os vértices, e outro, as arestas.

\section{Ambiente de Teste}
Os experimentos foram realizados usando um processador Intel i7 2600k acompanhado de 8 GiB de RAM. 
O sistema operacional utilizado foi Ubuntu Linux 16.10.

\section{Análise de Complexidade da \textit{Hollow Heap}}
A complexidade teórica das operaç\~oes da \textit{Hollow Heap} foram comparadas com os resultados de experimentos.
Conclui-se que a implementaç\~ao respeita as previs\~oes teóricas.

\subsection{Avaliaç\~ao das Operaç\~oes de Inserç\~ao}
Fixando-se o valor $n = 2^{27}-1$, adicionou-se $n$ elementos na \textit{Hollow Heap} com chaves no intervalo de $[n,1]$.
Para cada bloco com o intervalo de inserções $[2^{i-1},2^i - 1]$ para $i \in [0,26]$, foi medido o número de nodos criados
$C_i$, o número de melds $M_i$ e tempo de execução $T_i$. Para cada valor inserido na \textit{Hollow Heap}, um \textit{meld} é executado.
Um \textit{meld} é a união de duas \textit{heaps}. Onde um nodo único também é considerado uma heap. Com experimentos, verificou-se que 
o número de chamadas as rotinas de inserção e \textit{meld} foram chamadas uma vez para cada elemento inserido como era esperado.
A inserção de um elemento na heap tem custo $O(1)$ (\textit{meld} do elemento com a \textit{heap}). Logo, inserir $n$ nodos 
terá custo $O(n)$. Comparando o número de inserções com o número de \textit{melds}, percebe-se que eles são os mesmos indicando
que a implementaçao respeita a definição do algoritmo. Comparando-se o tempo de execuçao com o número de operações esperadas
pela complexidade, vê-se uma relação constante como pode ser vista na figura \ref{fig:insert}

Considerando as duas comparaç\~oes, conclui-se que a implementaç\~ao da inserç\~ao na \textit{Hollow Heap}
respeita a complexidade do algoritmo.


\begin{figure}[H]
\centering

\begin{tikzpicture}

\begin{axis}[
  title={},
  xlabel=$2^i$,
  ylabel=raz\~ao]
  ]
\addplot +[mark=o, color=red] table [x=2toi, y=OdivT, col sep=comma] {insert.csv};
\end{axis}
\end{tikzpicture}

\caption{Raz\~ao entre tempo de execuç\~ao e o custo esperado.}
\label{fig:insert}
\end{figure}	

\subsection{Avaliaç\~ao das Operaç\~oes de Decrescimento de Chaves}
Considerando-se uma variável $i$, $2^i -1$ chaves s\~ao inseridas com valor de $2^i +1$. Em seguida, $2^i$ chaves com valor $2^i +2$ s\~ao adicionadas.
Medindo-se o tempo, as chaves com valor, $2^i +2$ s\~ao atualizadas para valores decrescentes no intervalo $[2^i, 1]$. As operaç\~oes de atualizaç\~ao 
começam por $2^i$ e a cada operaç\~ao seguinte a esse valor é decrescido por 1. A complexidade para a operação de \textit{decrease key} é $O(1)$. Logo,
para atualizar as $2^i$ chaves, a complexidade é de $O(2^i)$. Ao comparar essa complexidade com o tempo de execuçao, nota-se a convergência da razão
desses valores para um valor como mostra a figura \ref{fig:update1}. Assim, pode-se concluir que o algoritmo escala como previsto.

\begin{figure}[H]
\centering
\begin{tikzpicture}
\begin{axis}[
  title={},
  xlabel=$2^i$,
  ylabel=raz\~ao]
  ]
\addplot +[mark=o, color=red] table [x=2toi, y=ratio, col sep=comma] {decreasekey.csv};
\end{axis}
\end{tikzpicture}
\caption{Mostra a raz\~ao entre o tempo de execuç\~ao e o custo teórico esperado}
\label{fig:update1}
\end{figure}

\subsection{Avaliaç\~ao das Operaç\~oes de Remoç\~ao Mínima}
Para um valor $i$, $n = 2^{i} - 1$ chaves com valor aleatórios s\~ao adicionados. Depois, $m = 2^{i - 1}$ chaves s\~ao removidas.
O tempo de execuç\~ao e o número de swaps foram medidos para cada i. A complexidade da operação de \textit{delete min} é de
$O(log~n)$ onde n é o número de nós na heap. Para remover $m$ nodos, a complexidade é $O(log(n) * m)$.
Na figura \ref{fig:delete1} é comparado esse custo com o tempo de execução.
A convergência a um valor indica que o tempo de execuç\~ao cresce conforme a
complexidade do algoritmo.

\begin{figure}[H]
\centering
\begin{tikzpicture}
\begin{axis}[
  title={},
  xlabel=$2^i-1$,
  ylabel=raz\~ao]
  ]
\addplot +[mark=o, color=red] table [x=size, y=ratio, col sep=comma] {deletemin.csv};
\end{axis}
\end{tikzpicture}
\caption{Mostra a raz\~ao entre o tempo de execuç\~ao e o custo teórico esperado}
\label{fig:delete1}
\end{figure}

\subsection{Comparando \textit{Binary Heap} e \textit{Hollow Heap}}
O tempo de execução para as operações descritas acima nos dois tipos de heaps foram comparados. Essas comparações estão plotadas nas figuras
\ref{fig:compare_insert}, \ref{fig:compare_decrease} e \ref{fig:compare_deletemin}. Com exceção das operações de \textit{delete min}, 
a \textit{Hollow Heap} teve melhor desempenho.

\begin{figure}[H]
\centering
\begin{tikzpicture}
\begin{axis}[
  title={},
  xlabel=$i$,
  ylabel=$Tempo$]
  ]
\addplot +[mark=none, color=red] table [x=i, y=Hollow, col sep=comma, mark=none, smooth] {comparison_insert.csv};
\addlegendentry{Hollow Heap}
\addplot +[mark=none, color=blue] table [x=i, y=Binary, col sep=comma, mark=none, smooth] {comparison_insert.csv};
\addlegendentry{Binary Heap}
\end{axis}
\end{tikzpicture}
\caption{Tempo de execução para testes de inserção}
\label{fig:compare_insert}
\end{figure}

\begin{figure}[H]
\centering
\begin{tikzpicture}
\begin{axis}[
  title={},
  xlabel=$i$,
  ylabel=$Tempo$]
  ]
\addplot +[mark=none, color=red] table [x=i, y=Hollow, col sep=comma, mark=none, smooth] {comparison_decrease.csv};
\addlegendentry{Hollow Heap}
\addplot +[mark=none, color=blue] table [x=i, y=Binary, col sep=comma, mark=none, smooth] {comparison_decrease.csv};
\addlegendentry{Binary Heap}
\end{axis}
\end{tikzpicture}
\caption{Tempo de execução para testes de \textit{decrease key}}
\label{fig:compare_decrease}
\end{figure}

\begin{figure}[H]
\centering
\begin{tikzpicture}
\begin{axis}[
  title={},
  xlabel=$i$,
  ylabel=$Tempo$]
  ]
\addplot +[mark=none, color=red] table [x=i, y=Hollow, col sep=comma, mark=none, smooth] {comparison_deletemin.csv};
\addlegendentry{Hollow Heap}
\addplot +[mark=none, color=blue] table [x=i, y=Binary, col sep=comma, mark=none, smooth] {comparison_deletemin.csv};
\addlegendentry{Binary Heap}
\end{axis}
\end{tikzpicture}
\caption{Tempo de execução para testes de \textit{delete min}}
\label{fig:compare_deletemin}
\end{figure}



\section{Análise de Complexidade do Algoritmo de Dijkstra}
\subsection{Variando o número de Arestas}
Fixando-se o número de vértices em $2^{15}$, variou-se o número de arestas entre $2^{23}$ e $2^{27}$. Esse intervalo
foi pequeno devido a restrições de memória no limite superior e o de precisão na medição de tempo no inferior.
A complexidade do algoritmo de Dijkstra é de $O(m + n*log(n))$. Comparando o tempo de execuçao com essa complexidade - 
através da divisão da complexidade esperada para o tamanho da instância pelo tempo - obteve-se a curva na figura 
\ref{fig:dij_vertex}. Nota-se uma convergência para um valor constante indicando que a implementação segue os limites
teóricos.
Os números de \textit{inserts} e \textit{delete mins} foram
menores ou iguais ao de vértices e o de \textit{decrease min}, menor ou igual ao de arestas. Uma regress\~ao
linear na figura \ref{fig:dij_vertex_linear_regression} sobre os dados obtidos mostrou que o tempo de execução
cresce exponencialmente de acordo com a soma do número de arestas.

\begin{figure}[H]
\centering

\begin{tikzpicture}

\begin{axis}[
  title={},
  xlabel=$m$,
  ylabel=$m + n*log(n)/T$
  ]
\addplot +[mark=none, color=red] table [x=m, y=ratio, col sep=comma, mark=none, smooth] {fix_vertex.csv};
\end{axis}
\end{tikzpicture}
\caption{Mostra a raz\~ao entre o custo teórico e o tempo de execução}
\label{fig:dij_vertex}
\end{figure}


\begin{figure}[H]
\centering

\begin{tikzpicture}

\begin{axis}[
  title={},
  xlabel=$log(m)$,
  ylabel=$log(time)$]
  \addplot +[mark=o, color=red,only marks] table [x=logn, y=logtime, col sep=comma] {fix_vertex_lin_regression.csv};
  \addplot +[mark=none, color=blue] table [x=logn, y=pred, col sep=comma] {fix_vertex_lin_regression.csv};
\end{axis}
\end{tikzpicture}
\caption{Regress\~ao linear. O número de vértices foi desconsiderado pois é constante.}
\label{fig:dij_vertex_linear_regression}
\end{figure}

\subsection{Variando o número de Vértices}
Fixou-se o número de arestas em $2^{20}$ e variou-se o número de vértices entre $2^{11}$ e $2^{18}$. A complexidade é a mesma do caso anterior: $O(m + n*log(n))$.
Comparando-se o tempo de processamento como no caso anterior, obteve-se os resultados mostrados na figura \ref{fig:dij_edges}. O gráfico mostra que o valor tende
a uma constante indicando que a implementação respeita a complexidade do algoritmo. Uma regress\~ao
linear na figura \ref{fig:dij_edge_lin_reg} sobre os dados obtidos mostrou que o tempo de execução
cresce exponencialmente de acordo com a soma do número de vértices.



\begin{figure}[H]
\centering
\begin{tikzpicture}
\begin{axis}[
  title={},
  xlabel=$m$,
  ylabel=$m + n*log(n)/T$]
  ]
\addplot +[mark=none, color=red] table [x=n, y=ratio, col sep=comma, mark=none, smooth] {fix_edges.csv};
\end{axis}
\end{tikzpicture}
\caption{Mostra a raz\~ao entre o custo teórico e o tempo de execução}
\label{fig:dij_edges}
\end{figure}

\begin{figure}[H]
\begin{tikzpicture}

\begin{axis}[
  title={},
  xlabel=$log(n)$,
  ylabel=$log(time)$]
  ]
\addplot +[mark=o, color=red,only marks] table [x=logn, y=logtime, col sep=comma] {fix_edge_lin_regression.csv};
\addplot +[mark=none, color=blue] table [x=logn, y=pred, col sep=comma] {fix_edge_lin_regression.csv};
\end{axis}
\end{tikzpicture}
\centering
\caption{Regress\~ao linear. O número de arestas foi desconsiderado pois é constante.}
\label{fig:dij_edge_lin_reg}
\end{figure}

\subsection{Comparando \textit{Binary Heap} e \textit{Hollow Heap}}
Os tempos de execução para o algoritmo de Dijkstra usando as duas heaps foram comparados. Essas comparações estão plotadas nas figuras
\ref{fig:compare_fix_vertex} e \ref{fig:compare_fix_edge}. Em ambos os casos a \textit{Hollow Heap} teve melhor desempenho.

\begin{figure}[H]
\centering
\begin{tikzpicture}
\begin{axis}[
  title={},
  xlabel=$i$,
  ylabel=$Tempo$]
  ]
\addplot +[mark=none, color=red] table [x=i, y=Hollow, col sep=comma, mark=none, smooth] {comparison_fix_vertex.csv};
\addlegendentry{Hollow Heap}
\addplot +[mark=none, color=blue] table [x=i, y=Binary, col sep=comma, mark=none, smooth] {comparison_fix_vertex.csv};
\addlegendentry{Binary Heap}
\end{axis}
\end{tikzpicture}
\caption{Tempo de execução para testes de inserção}
\label{fig:compare_fix_vertex}
\end{figure}

\begin{figure}[H]
\centering
\begin{tikzpicture}
\begin{axis}[
  title={},
  xlabel=$i$,
  ylabel=$Tempo$]
  ]
\addplot +[mark=none, color=red] table [x=i, y=Hollow, col sep=comma, mark=none, smooth] {comparison_fix_edge.csv};
\addlegendentry{Hollow Heap}
\addplot +[mark=none, color=blue] table [x=i, y=Binary, col sep=comma, mark=none, smooth] {comparison_fix_edge.csv};
\addlegendentry{Binary Heap}
\end{axis}
\end{tikzpicture}
\caption{Tempo de execução para testes de \textit{decrease key}}
\label{fig:compare_fix_edge}
\end{figure}




\section{Verificando a implementaç\~ao de Dijkstra com grandes instâncias}
Testou-se a implementaç\~ao do algoritmo com duas instâncias conforme recomendado no plano de teste. Comparou-se, também, com os resultados
obtidos com as \textit{Binary heaps} implementadas anteriormente. A comparação para os dois casos pode ser vista nas tabelas \ref{tab:hollow} e \ref{tab:nary}.
O caso usando \textit{Hollow Heaps} teve um tempo de execução menor em ambos os casos a custo de um consumo maior de memória.


\begin{table}[H]
  \centering
 \begin{tabular}{c | c c}
    rede & tempo de execução (s) & máximo de memória (MiB) \\
    \hline
    NY	    & 1.62 & 54.41 \\
    USA     & 141.81 & 5204 \\
 \end{tabular}
 \label{tab:hollow}
  \caption{Tempo de execução do algoritmo de Dijkstra usando Hollow Heaps}
\end{table}

\begin{table}[H]
\centering
 \begin{tabular}{c | c c}
    rede & tempo de execução (s) & máximo de memória (MiB) \\
    \hline
    NY	    & 1.83 & 42.67  \\
    USA     & 169.78 & 3555  \\
 \end{tabular}
 \label{tab:nary}
  \caption{Tempo de execução do algoritmo de Dijkstra usando Binary Heaps}
\end{table}




\section{Conclus\~ao}
Implementou-se a estrutura de dados \textit{Hollow Heap} assim como o algoritmo de Dijkstra usando essa estrutura. A implementação respeitou as previsões teóricas.
Ao calcular a regressão linear para os experimentos envolvendo o algoritmo de Dijkstra, obteve-se no entretanto uma comprovação de uma hipótese exponencial o que contraria
a complexidade do algoritmo. Comparou-se a performance da \textit{Hollow Heap} e da \textit{Binary Heap} nos vários experimentos. A primeira teve melhores tempos de execução
na maior parte dos casos. Essa melhora tem o custo de maior uso de memória.
\end{document}
